\documentclass{article}

\begin{document}

\title{Project plan}
\author{L Oswald}
\date{}

\maketitle

\section{Wednesday, 1st July 2015}

Summary of the science: the research problem/challenge I propose is to figure out how
to connect parametric vs free-form models.  As you know, with
SpaghettiLens we currently produce an ensemble of free-form
(pixellated) mass maps.  But most people work with parametrized
functional forms for the mass distribution.  It would be nice if
someone could name their favourite functional form, and the machine
could give them the parameter values which was close to the free-form
mass map.  I think a form of principal-component analysis will get us
there -- but it has not been tried yet!\newline 

\noindent Done so far:
\begin{itemize}
  \item Installed github, Spyder and LaTeX
  \item Written program to calculate eigenvectors and eigenvalues of moment of inertia tensor of data set. Set up parameterised model and use program to optimise the fit between model and data
  \item Write a summary of the physics involved so far (and ensure I understand it)
  \item Comment code so that it is easy to follow and make naming conventions more consistent so that I can work from the code more easily
  \item Write a summary of the next steps I think are required in both the physics and the programming
  \item Test out theories on the program to see how best to proceed and to highlight any problems (started)
  \item Maintain a log as I go along
\end{itemize}\newpage

\section{Thursday, 2nd July 2015}
Suggested steps to take:
\begin{itemize}
  \item Investigate chi-squared fit as an alternative to the one currently being used
  \item Continue investigating the ability of the program to resolve models with multiple parameters and see if it's possible to solve the runtime warning problem currently present
  \item Modify script to allow user to enter a chosen parametric form, first for a set number of parameters but then look into specifying the number of parameters too?
  \item Write up the summary of the physics so far in LaTeX, partly as it would be useful to have for report purposes later and partly to practise using LaTeX!
  \item Look into getting multiple data sets and seeing how to modify the code to handle that
  \item Make the output of the code more user-friendly eg. print out the parameters in a more readable format
\end{itemize}

\noindent Plan for the day:
\begin{enumerate}
  \item Work out how to do branching in github, set up a branch to experiment on
  \item Use said branch to practise Python on, working on getting the ouput more user-friendly (use Hello World book etc for help)
  \item Read up on principal component analysis to get ideas of how to improve the fit
  \item Lunch
  \item Continue with any of the above that need more work
  \item Work on improving the chi-squared model
  \item Start modifying the code to handle multiple data sets
  \item Summarise work done today
\end{enumerate}


\end{document}
