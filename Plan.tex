\documentclass{article}

\begin{document}

\title{Project plan}
\author{L Oswald}
\date{}

\maketitle

\section{Wednesday, 1st July 2015}

Summary of the science: the research problem/challenge I propose is to figure out how
to connect parametric vs free-form models.  As you know, with
SpaghettiLens we currently produce an ensemble of free-form
(pixellated) mass maps.  But most people work with parametrized
functional forms for the mass distribution.  It would be nice if
someone could name their favourite functional form, and the machine
could give them the parameter values which was close to the free-form
mass map.  I think a form of principal-component analysis will get us
there -- but it has not been tried yet!\newline 

\noindent Done so far:
\begin{itemize}
  \item Installed github, Spyder and LaTeX
  \item Written program to calculate eigenvectors and eigenvalues of moment of inertia tensor of data set. Set up parameterised model and use program to optimise the fit between model and data
\end{itemize}

\noindent To do today:
\begin{itemize}
  \item Write a summary of the physics involved so far (and ensure I understand it)
  \item Comment code so that it is easy to follow and make naming conventions more consistent so that I can work from the code more easily
  \item Write a summary of the next steps I think are required in both the physics and the programming
  \item Test out theories on the program to see how best to proceed and to highlight any problems
  \item Maintain a log as I go along
\end{itemize}

\end{document}
