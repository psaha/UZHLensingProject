\documentclass{article}

\begin{document}

\title{Project plan}
\author{L Oswald}
\date{}

\maketitle

\section{Wednesday, 1st July 2015}

Summary of the science: the research problem/challenge I propose is to figure out how
to connect parametric vs free-form models.  As you know, with
SpaghettiLens we currently produce an ensemble of free-form
(pixellated) mass maps.  But most people work with parametrized
functional forms for the mass distribution.  It would be nice if
someone could name their favourite functional form, and the machine
could give them the parameter values which was close to the free-form
mass map.  I think a form of principal-component analysis will get us
there -- but it has not been tried yet!\newline 

\noindent Done so far:
\begin{itemize}
  \item Installed github, Spyder and LaTeX
  \item Written program to calculate eigenvectors and eigenvalues of moment of inertia tensor of data set. Set up parameterised model and use program to optimise the fit between model and data
  \item Write a summary of the physics involved so far (and ensure I understand it)
  \item Comment code so that it is easy to follow and make naming conventions more consistent so that I can work from the code more easily
  \item Write a summary of the next steps I think are required in both the physics and the programming
  \item Test out theories on the program to see how best to proceed and to highlight any problems (started)
  \item Maintain a log as I go along
\end{itemize}\newpage

\section{Thursday, 2nd July 2015}
Suggested steps to take:
\begin{itemize}
  \item Investigate chi-squared fit as an alternative to the one currently being used
  \item Continue investigating the ability of the program to resolve models with multiple parameters and see if it's possible to solve the runtime warning problem currently present
  \item Modify script to allow user to enter a chosen parametric form, first for a set number of parameters but then look into specifying the number of parameters too?
  \item Write up the summary of the physics so far in LaTeX, partly as it would be useful to have for report purposes later and partly to practise using LaTeX!
  \item Look into getting multiple data sets and seeing how to modify the code to handle that
  \item Make the output of the code more user-friendly eg. print out the parameters in a more readable format
\end{itemize}

\noindent Plan for the day:
\begin{enumerate}
  \item Work out how to do branching in github, set up a branch to experiment on
  \item Use said branch to practise Python on, working on getting the ouput more user-friendly (use Hello World book etc for help)
  \item Read up on principal component analysis to get ideas of how to improve the fit
  \item Work on improving the chi-squared model
  \item Start modifying the code to handle multiple data sets
  \item Summarise work done today
\end{enumerate}

\noindent Done today:
\begin{itemize}
  \item Set up secondary branch, tested out small edits on it
  \item Worked on getting the code output more user friendly e.g. graph labels, outputting parameters etc
  \item Learned how to do argument parsing, made little program which took in arguments from command line
  \item Read up on Python and PCA
  \item Downloaded another lens to work on
\end{itemize}\newpage

\section{Friday, 3rd July 2015}
\noindent Suggested steps to take:
\begin{itemize}
  \item Continue working on models to find a way that it can produce parameterised models that are consistently good, using both lenses now
  \item Chi-squared fit
  \item Finish reading up on PCA and write up a summary of the physics in LaTeX
  \item Find out more about modules in Python and look into turning this program into a module that can be used by others
\end{itemize}

\noindent Plan for the day:
\begin{enumerate}
  \item Read tutorial on PCA and chapter on modules
  \item Start writing up physics summary (in style of paper intro?)
  \item Play around with chi-squared fit
  \item Play around with new lens
  \item General playing with coding: see what works to get consistent good fits and improve the output of the code further
\end{enumerate}

Done today:
\begin{itemize}
  \item All of the above except for the physics summary
  \item New parametrised model from equation for gravitational potential of an elliptical lens
 \end{itemize}\newpage

\section{Monday, 6th July 2015}
\noindent Plan for the day:
\begin{enumerate}
  \item Type up quick summary of PCA stuff, don't worry about whether it's good or not as that can be done later
  \item Fiddle with code for a bit, see if you can make it more efficient like in the Python tutorial
  \item Basically spend the day doing a combo of playing with the model and writing up the results of the playing!
\end{enumerate}

\noindent Done today:
\begin{itemize}
  \item Rough summary of PCA stuff
  \item Fiddled with code, nothing from Python PCA guide worked
  \item Changed code to find the points on the MoI ellipse that are closest to the parameterised form by adding projections (of the parameterised form along the eigenvectors) to the mean, and plotted this against the parameterised model instead of plotting the mean
  \item Used SpaghettiLens to make a new model of the simulated lens (gribbles) and tested it with the new code
  \item Watched the SpaghettiLens tutorial and started thinking about ways it could be improved
\end{itemize}

\noindent To do:
\begin{itemize}
  \item Improve spaghettilens model gribbles
  \item Try to get original parameters used to make the model back out the other end: aim to do this test tomorrow
  \item Spaghetti lens tutorial: write down questions/comments
\end{itemize}\newpage

\section{Tuesday, 7th July 2015}
\noindent Plan for the day:
\begin{enumerate}
  \item Questions/comments on SpaghettiLens tutorial (done)
  \item Plot contours with colour bars (done)
  \item Experiment with new model (done)
  \item Experiment with improved gribbles model
  \item Tidy up PCA summary
  \item Continue log
\end{enumerate}

\noindent Done today:
\begin{itemize}
  \item Comments on SpagLens tutorial
  \item Colour-filled contour plotting
  \item Experiment with new model
  \item Add in unit-conversion to arcseconds for one lens
  \item Add fudge factor (problems currently)
  \item Started tidying PCA summary
  \item Commented/tidied code
\end{itemize}



\end{document}
